\documentclass{article}
\usepackage[english]{babel}
\usepackage{graphicx,times,amsmath,colortbl, psfrag}
\usepackage[ruled, vlined, english, boxed, linesnumbered, lined]{algorithm2e}
\usepackage[letterpaper,top=2cm,bottom=2cm,left=3cm,right=3cm,marginparwidth=1.75cm]{geometry}
\usepackage{amsmath}
\usepackage{listings}
\usepackage{algorithm}
\usepackage{algorithmic}
\usepackage{algpseudocode}
\usepackage{graphicx}
\usepackage[colorlinks=true, allcolors=blue]{hyperref}
\usepackage{algpseudocode} 
\usepackage{multirow}
\usepackage {amsfonts}
\begin{document}
\begin{titlepage}
\centering

{\bfseries\LARGE Instituto Polit\'ecnico Nacional \par}
\vspace{1cm}
{\scshape\Large Escuela Superiror de Computaci\'on \par}
\vspace{3cm}
{\scshape\Huge  PILAS \par}
\vspace{3cm}
{\itshape\Large Pr\'actica 6 \par}
\vfill
{\Large Alumno: \par}
{\Large Alcántara Covarrubias Erik \par}
{\Large Profesor: \par}
{\Large Juarez Martinez Genaro \par}
{\Large Grupo: 4CM6\par}
\vfill
\end{titlepage}

\section{Introducci\'on}
Fundamentalmente, el autómata a pila es un autómata finito no determinista con transiciones- $\epsilon$ y una capacidad adicional: una pila en la que se puede almacenar una cadena de “símbolos de pila”. La presencia de una pila significa que, a diferencia del autómata finito, el autómata a pila puede “recordar” una cantidad infinita de información.\newline \newline Sin embargo, a diferencia de las computadoras de propósito general, que también tienen la capacidad de recordar una cantidad arbitrariamente grande de información, el autómata a pila solo puede acceder a la información disponible en su pila de acuerdo con la forma de manipular una pila FIFO (first-in-first-out way, primero en entrar primero en salir).\newline\newline Así, existen lenguajes que podrían ser reconocidos por determinados programas informáticos, pero no por cualquier autómata a pila. De hecho, los autómatas a pila reconocen todos los lenguajes independientes del contexto y solo estos.
\section{Marco teórico}
Los autómatas de pila, en forma similar a como se usan los autómatas finitos, también se pueden utilizar para aceptar cadenas de un lenguaje definido sobre un alfabeto $A$. \newline Los autómatas de pila pueden aceptar lenguajes que no pueden aceptar los autómatas finitos. Un autómata de pila cuenta con una cinta de entrada y un mecanismo de control que puede encontrarse en uno de entre un número finito de estados. Uno de estos estados se designa como estado inicial, y además algunos estados se llaman de aceptación o finales.\newline A diferencia de los autómatas finitos, los autómatas de pila cuentan con una memoria auxiliar llamada pila. Los símbolos (llamados símbolos de pila) pueden ser insertados o extraídos de la pila, de acuerdo con el manejo $last-in-first-out (LIFO)$.\newline\newline Las transiciones entre los estados que ejecutan los autómatas de pila dependen de los símbolos de entrada y de los símbolos de la pila. El autómata acepta una cadena x si la secuencia de transiciones, comenzando en estado inicial y con pila vacía, conduce a un estado final, después de leer toda la cadena x.\newline \newline
La notación formal de un autómata a pila incluye siete componentes. Escribimos la especificación de un autómata a pila P de la forma siguiente:
\begin{center}
    $P=(Q,\Sigma,\Gamma,\delta,q_0,Z_0,F)$
\end{center}
El significado de cada uno de los componentes es el siguiente:
\begin{itemize}
    \item $Q$: Un conjunto finito de estados, como los estados de un autómata finito.
    \item $\Sigma$: Un conjunto finito de símbolos de entrada, también análogo al componente correspondiente de un autómata finito.
    \item $\Gamma$: Un alfabeto de pila finito. Este componente, que no tiene análogo en los autómatas finitos, es el conjunto de símbolos que pueden introducirse en la pila.
    \item $\delta$: La función de transición. Como en el autómata finito, $\delta$ controla el comportamiento del autómata. Formalmente, $\delta$ toma como argumento $\delta(q,a,X)$, donde:
    \begin{enumerate}
        \item $q$ es un estado de $Q$.
        \item $a$ es cualquier símbolo de entrada de $\Sigma$ o $a=\epsilon$, la cadena vacía, que se supone que no es un símbolo de entrada.
        \item $X$ es un símbolo de la pila, es decir, pertenece a $\Gamma$.
    \end{enumerate}
    La salida de $\delta$ es un conjunto finito de pares $(p,\gamma)$, donde $p$ es el nuevo estado y $\gamma$ es la cadena de símbolos de la pila que reemplaza $X$ en la parte superior de la pila.
    \item $q_0$: El estado inicial. El autómata a pila se encuentra en este estado antes de realizar ninguna transición.
    \item $Z_0$: El símbolo inicial. Inicialmente, la pila dels autómata a pila consta de una instancia de este símbolo y de nada más.
    \item $F$: El conjunto de estados de aceptación o estados finales.
\end{itemize}

\section{Explicaci\'on del problema e implementaci\'on del algoritmo}
\subsection{Problema a resolver}
Programar un autómata de pila que sirva para reconocer el lenguaje libre de contexto ${0^n 1^n | n >= 1}$.\newline 
\newline
Adicionalmente, el programa debe de contar con las siguientes características:
\begin{enumerate}
    \item La cadena puede ser ingresada por el usuario o automáticamente. Si es aleatoriamente, la cadena no podrá ser mayor a 100,000 caracteres.
    \item Mandar a un archivo y en pantalla la evaluación del autómata a través de descripciones instantáneas (IDs).
    \item Animar el autómata de pila, solo si la cadena es menor igual a 10 caracteres.
\end{enumerate}

\subsection{Implementaci\'on}
\begin{lstlisting}
    import random
import time
import os
from time import sleep

# It's a stack
class Pila(object):
    def __init__(self):
        self.largo = -1
        self.espacios = []

    def final(self):
        """
        If the length of the list is -1, then the list is empty
        :return: The final method returns a boolean value.
        """
        if self.largo == -1:
            return True
        else:
            return False

    def extraer(self):
        """
        It returns the last element of the list, and then removes it from the list
        :return: The value of the last element in the list.
        """
        if self.final():
            return 'e'
        else:
            valor = self.espacios[self.largo]
            self.largo -= 1
            return valor

    def insertar(self, elemento):
        """
        It inserts an element into the array.
        
        :param elemento: The element to be inserted
        """
        self.largo += 1
        self.espacios[self.largo:] = [elemento]

    def revelar(self):
        """
        It takes the length of the string, and then iterates through the string, adding each character to
        a new string, and then returns the new string
        :return: The string of the spaces in the list.
        """
        i = self.largo
        cadena = ''
        while(i>-1):
            cadena += self.espacios[i]
            i -= 1
        return cadena

def DES(cadena, decision):
    """
    It takes a string and a boolean as arguments, and if the boolean is true, it will print the string
    in a way that makes it look like a stack is being used to process the string
    
    :param cadena: The string to be tested
    :param decision: True or False, if True, the program will show the animation, if False, it will show
    the steps of the program
    """
    pila = Pila()
    archivo = open('Practica6/HISPILA.txt', 'w')
    pila.insertar('Zo')
    estado = 'q'
    auxiliar = cadena
    cadena = cadena + ' '
    archivo.write('La cadena es: ' + auxiliar + '\n')
    for simbolo in cadena:
        if auxiliar == '':
            auxiliar = 'e'
        if decision:
            time.sleep(1)
            ANIMACION(estado, auxiliar, pila)
        else:
            print('(%s, %s, %s)' %(estado, auxiliar, pila.revelar()), end='')
        archivo.write('(%s, %s, %s)' %(estado, auxiliar, pila.revelar()))
        if estado == 'q':
            if simbolo == '0':
                pila.insertar('X')
            elif simbolo == '1':
                if pila.extraer() == 'Zo':
                    pila.insertar('Zo')
                    break
                estado = 'p'
            else:
                estado='q'
                break
        elif estado == 'p':
            if simbolo == '1':
                if pila.extraer() == 'Zo':
                    estado = 'f'
                    pila.insertar('Zo')
                    break
            elif simbolo == '0':
                pila.insertar('X')
                auxiliar = auxiliar[1:]
                break
            elif simbolo == ' ':
                estado = 'f'
            else:
                break
        auxiliar = auxiliar[1:]
        archivo.write('->')

    if auxiliar == '':
        auxiliar = 'e'
    if (pila.revelar() == 'Zo') and auxiliar == 'e' and estado=='f':
        if decision:
            time.sleep(1)
            ANIMACION(estado, auxiliar, pila)
        else:
            print('(%s, %s, %s)' %(estado, auxiliar, pila.revelar()))
            print('\n')
        archivo.write('(%s, %s, %s)' %(estado, auxiliar, pila.revelar()))
        print('Esta cadena es valida')
        archivo.write('\nEsta cadena es valida')
    else:
        print('Esta cadena no es valida')
        archivo.write('\nEsta cadena no es valida')
    archivo.close()

def ANIMACION(estado, cadena_aux, stack):
    """
    It clears the screen, prints the current state, the current string, and the current stack, and then
    waits for 0.9 seconds
    
    :param estado: The current state of the automaton
    :param cadena_aux: The string that is being processed
    :param stack: is the stack that is used in the program
    """
    
    pila = 'Zo'    
    if stack.revelar() != '':
        pila = stack.revelar()
    if os.name =="nt": 
        os.system("cls") 
    else: 
        os.system("clear")
        
    print("\n")
    print(cadena_aux + " -> " +estado+" -> " + pila)
    sleep(0.9)

if __name__ == "__main__":
    
    while True:
        
        print("\n*****PROGRAMA 6: PILA*****")
        print("1.-Colocar la cadena")
        print("2.-Cadena aleatoria")
        print("3.-Salir")
        OP = int(input("Elija una opcion: "))
            
        if OP == 1:
            cadena = input("Escribe el numero binario: ")
        elif OP ==2:
            i = 0
            LONGITUD = random.randint(1, 100000)
            cadena = ''
            while i < LONGITUD:
                cadena += random.choice(['0', '1'])
                i += 1   
        elif OP ==3:
            print("Adios!!\n")
            exit()
        else:
            print("No existe esa opcion\n")
            break
        
        print("El numero es: ", cadena)
        largo = len(cadena)
        if largo <= 10:
            animacion = True
        else:
            print("No se puede realizar la animacion\n")
            animacion = False
            
        DES(cadena, animacion)
\end{lstlisting}

\subsection{Capturas de Resultados}

\section{Conclusi\'on}

\section{Bibliograf\'ia}
\begin{itemize}
    \item Introduction to Automata Theory, Languages, and Computation
    John E. Hopcroft, Rajeev Motwani, Jeffrey D. Ullman
    Pearson/Addison Wesley, 2nd edition
    \item Software Foundation, P. (2022, 13 enero). 3.10.2 Documentation. Documentación de Python. Recuperado 15 de febrero de 2022, de https://docs.python.org/es/3/
\end{itemize}
\end{document}

